\documentclass[12pt]{book}
\usepackage{amsmath}

\begin{document}

\chapter{Programming and Preparation}

\section{Program Output} \label{sec:program_output}
\section{Variables, Expressions, and Assignment} \label{sec:variables_expressions_assignment}
\section{The Use of \#define and \#include} \label{sec:use_define_include}
\section{The Use of \texttt{printf()} and \texttt{scanf()}} \label{sec:use_printf_scanf}
\section{Flow of Control} \label{sec:flow_of_control}
\section{Functions} \label{sec:functions}
\subsection{Call-by-Value} \label{sec:call_by_value}
\section{Arrays, Strings, and Pointers} \label{sec:arrays_strings_pointers}
\subsection{Arrays} \label{sec:arrays}
\subsection{Strings} \label{sec:strings}
\subsection{Pointers} \label{sec:pointers}
\section{Files} \label{sec:files}
\section{Operating System Considerations} \label{sec:os_considerations}

\chapter{Writing and Running a C Program}

\section{Interrupting a Program} \label{sec:interrupting_program}
\section{Typing an End-of-file Signal} \label{sec:end_of_file_signal}
\section{Redirection of the Input and the Output} \label{sec:redirection}

\chapter{Lexical Elements, Operators, and the C System}

\section{Characters and Lexical Elements} \label{sec:characters_lexical_elements}
\section{Syntax Rules} \label{sec:syntax_rules}
\section{Comments} \label{sec:comments}
\section{Keywords} \label{sec:keywords}
\section{Identifiers} \label{sec:identifiers}
\section{Constants} \label{sec:constants}
\section{String Constants} \label{sec:string_constants}
\section{Operators and Punctuators} \label{sec:operators_punctuators}
\section{Precedence and Associativity of Operators} \label{sec:precedence_associativity}
\section{Increment and Decrement Operators} \label{sec:increment_decrement_operators}
\section{Assignment Operators} \label{sec:assignment_operators}
\section{An Example: Computing Powers of 2} \label{sec:computing_powers_of_2}
\section{The C System} \label{sec:c_system}
\section{The Preprocessor} \label{sec:preprocessor}
\section{The Standard Library} \label{sec:standard_library}

\chapter{The Fundamental Data Types}

\section{Declarations, Expressions, and Assignment} \label{sec:declarations_expressions_assignment}
\section{The Fundamental Data Types} \label{sec:fundamental_data_types}
\section{Characters and the Data Type \texttt{char}} \label{sec:char_data_type}
\section{The Data Type \texttt{int}} \label{sec:int_data_type}
\section{The Integral Types: \texttt{short}, \texttt{long}, and \texttt{unsigned}} \label{sec:integral_types}
\section{The Floating Types} \label{sec:floating_types}
\section{The Use of \texttt{typedef}} \label{sec:use_of_typedef}
\section{The \texttt{sizeof} Operator} \label{sec:sizeof_operator}
\section{The Use of \texttt{getchar()} and \texttt{putchar()}} \label{sec:getchar_putchar}
\section{Mathematical Functions} \label{sec:mathematical_functions}
\section{The Use of \texttt{abs()} and \texttt{fabs()}} \label{sec:abs_fabs}
\section{UNIX and the Mathematics Library} \label{sec:unix_math_library}

\section{Conversions and Casts}
\section{The Integral Promotions}
\section{The Usual Arithmetic Conversions}
\section{Casts}
\section{Hexadecimal and Octal Constants}
\section{Summary}
\section{Exercises}

\chapter{Flow of Control}

\section{Relational, Equality, and Logical Operators}
\subsection{Relational Operators and Expressions}
\subsection{Equality Operators and Expressions}
\subsection{Logical Operators and Expressions}
\subsection{Short-circuit Evaluation}
\section{The Compound Statement}
\section{The Expression and Empty Statement}
\section{The \texttt{if} and the \texttt{if-else} Statements}
\section{The \texttt{while} Statement}
\section{The \texttt{for} Statement}
\subsection{An Example: Boolean Variables}
\section{The Comma Operator}
\section{The \texttt{do} Statement}
\subsection{An Example: Fibonacci Numbers}
\section{The \texttt{goto} Statement}
\section{The \texttt{break} and \texttt{continue} Statements}
\section{The \texttt{switch} Statement}
\section{The Conditional Operator}
\section{Summary}
\section{Exercises}

\chapter{Functions}

\section{Function Definition}
\section{The \texttt{return} Statement}
\section{Function Prototypes}
\section{Function Prototypes in C++}
\section{An Example: Creating a Table of Powers}
\section{Function Declarations from the Compiler's Viewpoint}
\section{Limitations}
\section{An Alternate Style for Function Definition Order}
\section{Function Invocation and Call-by-Value}
\section{Summary}
\section{Exercises}

\section{Developing a Large Program}

\section{What Constitutes a Large Program?}
\section{Using Assertions}
\section{Scope Rules}
\section{Parallel and Nested Blocks}
\section{Using a Block for Debugging}
\section{Storage Classes}
\subsection{The Storage Class \texttt{auto}}
\subsection{The Storage Class \texttt{extern}}
\subsection{The Storage Class \texttt{register}}
\subsection{The Storage Class \texttt{static}}
\subsection{Static External Variables}
\subsection{Default Initialization}
\section{Recursion}
\section{Efficiency Considerations}
\subsection{An Example: The Towers of Hanoi}
\section{Summary}
\section{Exercises}

\chapter{Arrays, Pointers, and Strings}

\section{One-dimensional Arrays}
\subsection{Initialization}
\subsection{Subscripting}
\section{Pointers}
\section{Call-by-Reference}
\section{The Relationship Between Arrays and Pointers}
\section{Pointer Arithmetic and Element Size}
\section{Arrays as Function Arguments}
\subsection{An Example: Bubble Sort}
\section{Dynamic Memory Allocation With \texttt{calloc()} and \texttt{malloc()}}
\section{Offsetting the Pointer}
\subsection{An Example: Merge and Merge Sort}
\section{Strings}
\section{String-Handling Functions in the Standard Library}
\section{Multidimensional Arrays}
\subsection{Two-dimensional Arrays}
\subsection{The Storage Mapping Function}
\subsection{Three-dimensional Arrays}
\subsection{Initialization}
\subsection{The Use of \texttt{typedef}}
\section{Arrays of Pointers}
\section{Arguments to \texttt{main()}}
\section{Ragged Arrays}
\section{Functions as Arguments}
\section{Functions as Formal Parameters in Function Prototypes}
\subsection{An Example: Using Bisection to Find the Root of a Function}
\section{The Kepler Equation}
\section{Arrays of Pointers to Function}
\section{The Type Qualifiers \texttt{const} and \texttt{volatile}}
\section{Summary}
\section{Exercises}

\chapter{Bitwise Operators and Enumeration Types}

\section{Bitwise Operators and Expressions}
\subsection{Bitwise Complement}
\subsection{Two’s Complement}
\subsection{Bitwise Binary Logical Operators}
\subsection{Left and Right Shift Operators}
\section{Masks}
\section{Software Tools: Printing an \texttt{int} Bitwise}
\section{Packing and Unpacking}
\section{Multibyte Character Constants}
\section{Enumeration Types}
\section{An Example: The Game of Paper, Rock, Scissors}
\section{Summary}
\section{Exercises}

\chapter{The Preprocessor}

\section{The Use of \#include}
\section{The Use of \#define}
\section{Syntactic Sugar}
\section{Macros with Arguments}
\section{The Type Definitions and Macros in \texttt{stddef.h}}
\section{An Example: Sorting with \texttt{qsort()}}
\section{An Example: Macros with Arguments}
\section{The Macros in \texttt{stdio.h} and \texttt{ctype.h}}
\section{Conditional Compilation}
\section{The Predefined Macros}
\section{The Operators \# and \#\#}
\section{The \texttt{assert()} Macro}
\section{The Use of \#error and \#pragma}
\section{Line Numbers}
\section{Summary}
\section{Exercises}

\chapter{Structures and Unions}

\section{Structures}
\subsection{Accessing Members of a Structure}
\subsection{Operator Precedence and Associativity: A Final Look}
\subsection{Using Structures with Functions}
\subsection{Initialization of Structures}
\subsection{An Example: Playing Poker}
\section{Unions}
\section{Bit Fields}
\section{An Example: Accessing Bits and Bytes}
\section{The ADT Stack}
\section{Summary}
\section{Exercises}

\chapter{Structures and List Processing}

\section{Self-referential Structures}
\section{Linear Linked Lists}
\section{Storage Allocation}
\section{List Operations}
\section{Some List Processing Functions}
\section{Insertion}
\section{Deletion}
\section{Stacks}
\subsection{An Example: Polish Notation and Stack Evaluation}
\section{Queues}
\section{Binary Trees}
\section{Binary Tree Traversal}
\section{Creating Trees}
\section{General Linked Lists}
\section{Traversal}
\section{The Use of \texttt{calloc()} and Building Trees}
\section{Summary}
\section{Exercises}

\chapter{Input/Output and the Operating System}

\section{The Output Function \texttt{printf()}}
\section{The Input Function \texttt{scanf()}}
\section{The Functions \texttt{fprintf()}, \texttt{fscanf()}, \texttt{sprintf()}, and \texttt{sscanf()}}
\section{The Functions \texttt{fopen()} and \texttt{fclose()}}
\subsection{An Example: Double Spacing a File}
\section{Using Temporary Files and Graceful Functions}
\section{Accessing a File Randomly}
\section{File Descriptor Input/Output}
\section{File Access Permissions}
\section{Executing Commands from Within a C Program}
\section{Using Pipes from Within a C Program}
\section{Environment Variables}
\section{The C Compiler}
\section{Using the Profiler}
\section{Libraries}
\section{How to Time C Code}
\section{The Use of \texttt{make}}
\section{The Use of \texttt{touch}}
\section{Other Useful Tools}
\section{Summary}
\section{Exercises}

\chapter{Advanced Applications}

\section{Creating a Concurrent Process with \texttt{fork()}}
\section{Overlaying a Process: the \texttt{exec...()} Family}
\section{Using the \texttt{spawn...()} Family}
\section{Interprocess Communication Using \texttt{pipe()}}
\section{Signals}
\subsection{An Example: The Dining Philosophers}
\section{Dynamic Allocation of Matrices}
\section{Why Arrays of Arrays Are Inadequate}
\section{Building Matrices with Arrays of Pointers}
\section{Adjusting the Subscript Range}
\section{Allocating All the Memory at Once}
\section{Returning the Status}
\section{Summary}
\section{Exercises}

\section{Moving from C to C++}

\section{Output}
\section{Input}
\section{Functions}
\section{Classes and Abstract Data Types}
\section{Overloading}
\section{Constructors and Destructors}
\section{Object-oriented Programming and Inheritance}
\section{Polymorphism}
\section{Templates}
\section{C++ Exceptions}
\section{Benefits of Object-oriented Programming}

\end{document}
